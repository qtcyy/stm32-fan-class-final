\section{程序设计}

\subsection{系统初始化模块}

\begin{lstlisting}
    /****************************************
    *          系统初始化模块开始           *
    ****************************************/

   char buf[30] = {0};
   float current_temp;
   float current_humi;

   // 硬件模块初始化
   delay_init(168);
   key_init();
   key_interrupt_init();
   lcd_init(FAN1);
   infrared_init();
   rgb_init();
   buzzer_init();
   led_init();
   usart_init(115200);
   htu21d_init();
   fan_pwm_init(50000 - 1, 336 - 1);

   /****************************************
    *        LCD初始化显示模块开始         *
    ****************************************/

   // LCD界面初始化显示 - 显示基本标签和初始值
   sprintf(buf, "Fan PWM:%3d%%", fan_pwm[count]);
   LCDDrawFont16_Next(get_centered_x(buf), 20 + 10 * 7, 4, 320, buf, 0x0000,
                      0xffff); // 显示初始PWM值
   LCDDrawFont16_Next(get_centered_x("Mode:"), 30 + 10 * 8, 4, 320,
                      "Mode:", 0x0000, 0xffff); // 显示模式标签
   LCDDrawFont16_Next(get_centered_x("Temp:"), 30 + 10 * 10, 4, 320,
                      "Temp:", 0x0000, 0xffff); // 显示温度标签
   LCDDrawFont16_Next(get_centered_x("Humi:"), 30 + 10 * 12, 4, 320,
                      "Humi:", 0x0000, 0xffff); // 显示湿度标签
\end{lstlisting}

\subsection{温、湿度检测模块}

\begin{lstlisting}
    /*********************************************************
    * 功能:读取温度
    * 参数:无
    * 返回:-1/float t -- 处理后的温度值
    **********************************************************/
   float htu21d_t(void) {
     char cmd = 0xf3;
     char dat[4];
     i2c_write(HTU21D_ADDR, &cmd, 1);
     delay_ms(50);
     if (i2c_read(HTU21D_ADDR, dat, 2) == 2) {
       if ((dat[1] & 0x02) == 0) {
         float t =
             -46.85f + 175.72f * ((dat[0] << 8 | dat[1]) & 0xfffc) / (1 << 16);
         return t;
       }
     }
     return -1;
   }

   /************************************************************
    * 名称:htu21d_h()
    * 功能:读取湿度
    * 参数:无
    * 返回:-1/float h -- 处理后的湿度值
    ************************************************************/
   float htu21d_h(void) {
     char cmd = 0xf5;
     char dat[4];

     i2c_write(HTU21D_ADDR, &cmd, 1);
     delay_ms(50);
     if (i2c_read(HTU21D_ADDR, dat, 2) == 2) {
       if ((dat[1] & 0x02) == 0x02) {
         float h = -6 + 125 * ((dat[0] << 8 | dat[1]) & 0xfffc) / (1 << 16);
         return h;
       }
     }
     return -1;
   }
\end{lstlisting}

\subsection{UART通信模块}

\begin{lstlisting}
    /****************************************
    *          UART通信模块开始            *
    ****************************************/

   void check_uart_command(void) {
     // 检查串口接收缓冲区是否有足够的数据
     if (Usart_len >= 4) {
       // 检查是否接收到"Auto"命令
       if (strncmp((char *)USART_RX_BUF, "Auto", 4) == 0) {
         mode = AUTO_MODE;
         led_control(D2);
         printf("Mode switched to Auto\r\n");
         update_display(htu21d_t(), htu21d_h());
       }
       // 检查是否接收到"Manual"命令
       else if (strncmp((char *)USART_RX_BUF, "Manual", 6) == 0) {
         mode = MANUAL_MODE;
         led_control(D1);
         printf("Mode switched to Manual\r\n");
         update_display(htu21d_t(), htu21d_h());
       }
       // 清空串口接收缓冲区
       clean_usart();
     }
   }
\end{lstlisting}

\subsection{风扇模块}

\begin{lstlisting}
    /************************************************************
    * 功能:风扇传感器初始化
    * 参数:无
    * 返回:无
    *************************************************************/
   void fan_init(void) {
     GPIO_InitTypeDef GPIO_InitStructure; // 定义一个GPIO_InitTypeDef类型的结构体
     RCC_AHB1PeriphClockCmd(RCC_AHB1Periph_GPIOE,
                            ENABLE); // 开启风扇传感器相关的GPIO外设时钟

     GPIO_InitStructure.GPIO_Pin = GPIO_Pin_5;        // 选择要控制的GPIO引脚
     GPIO_InitStructure.GPIO_OType = GPIO_OType_PP;   // 设置引脚的输出类型为推挽
     GPIO_InitStructure.GPIO_Mode = GPIO_Mode_OUT;    // 设置引脚模式为输出模式
     GPIO_InitStructure.GPIO_PuPd = GPIO_PuPd_DOWN;   // 设置引脚为下拉模式
     GPIO_InitStructure.GPIO_Speed = GPIO_Speed_2MHz; // 设置引脚速率为2MHz

     GPIO_Init(GPIOE, &GPIO_InitStructure); // 初始化GPIO配置
     GPIO_ResetBits(GPIOE, GPIO_Pin_5);
   }

   /**************************************************************
    * 名称:void fan_control(unsigned char cmd)
    * 功能:风扇控制驱动
    * 参数:控制命令
    * 返回:无
    **************************************************************/
   void fan_control(unsigned char cmd) {
     if (cmd & 0x01)
       GPIO_SetBits(GPIOE, GPIO_Pin_5);
     else
       GPIO_ResetBits(GPIOE, GPIO_Pin_5);
   }

   static u32 cycle; // 这个值不要小于100,否则占空比不准确
   /***************************************************************
    * 名称:fan_pwm_init()
    * 功能:风扇传感器PWM初始化  PE5 连接 TIM9——CH1  16位定时器
    * 参数:arr:自动重装值  psc:时钟预分频数
    * 返回:无
    ***************************************************************/
   void fan_pwm_init(u32 arr, u32 psc) {
     cycle = arr + 1; // 用来计算占空比。需要加1
     // 此部分需手动修改IO口设置
     GPIO_InitTypeDef GPIO_InitStructure = {0};
     TIM_TimeBaseInitTypeDef TIM_TimeBaseStructure = {0};
     TIM_OCInitTypeDef TIM_OCInitStructure = {0};

     RCC_APB2PeriphClockCmd(RCC_APB2Periph_TIM9, ENABLE);  // TIM9时钟使能
     RCC_AHB1PeriphClockCmd(RCC_AHB1Periph_GPIOE, ENABLE); // 使能PORTF时钟

     GPIO_PinAFConfig(GPIOE, GPIO_PinSource5, GPIO_AF_TIM9); // GPIOE5
                                                             // 复用为定时器9

     GPIO_InitStructure.GPIO_Pin = GPIO_Pin_5;          // GPIOE5
     GPIO_InitStructure.GPIO_Mode = GPIO_Mode_AF;       // 复用功能
     GPIO_InitStructure.GPIO_Speed = GPIO_Speed_100MHz; // 速度100MHz
     GPIO_InitStructure.GPIO_OType = GPIO_OType_PP;     // 推挽复用输出
     GPIO_InitStructure.GPIO_PuPd = GPIO_PuPd_UP;       // 上拉
     GPIO_Init(GPIOE, &GPIO_InitStructure);             // 初始化PE5

     TIM_TimeBaseStructure.TIM_Prescaler =
         psc; // 定时器分频,定时器9挂载到APB2,为168MHZ,如果这里分频到1MHZ,设置为167
     TIM_TimeBaseStructure.TIM_CounterMode = TIM_CounterMode_Up; // 向上计数模式
     TIM_TimeBaseStructure.TIM_Period = arr;                     // 自动重装载值
     TIM_TimeBaseStructure.TIM_ClockDivision = TIM_CKD_DIV1;
     TIM_TimeBaseInit(TIM9, &TIM_TimeBaseStructure); // 初始化定时器9

     // 初始化TIM9 Channel1 PWM模式
     TIM_OCInitStructure.TIM_OCMode =
         TIM_OCMode_PWM1; // 选择定时器模式:TIM脉冲宽度调制模式
     TIM_OCInitStructure.TIM_OutputState = TIM_OutputState_Enable; // 比较输出使能
     TIM_OCInitStructure.TIM_OCPolarity = TIM_OCPolarity_High;     // 输出极性
     TIM_OC1Init(TIM9, &TIM_OCInitStructure);                      // 初始化通道1

     TIM_OC1PreloadConfig(TIM9,
                          TIM_OCPreload_Enable); // 使能TIM在CCR1上的预装载寄存器

     TIM_ARRPreloadConfig(TIM9, ENABLE); // ARPE使能
     TIM_Cmd(TIM9, ENABLE);              // 使能TIM
   }

   /******************************************************************
    * 名称:fan_pwm_control
    * 功能:风扇PWM驱动控制
    * 参数:pwm 占空比 0-100
    * 返回:无
    ******************************************************************/
   void fan_pwm_control(uint32_t pwm) {
     uint32_t _pwm = cycle / 100 * pwm;
     TIM_SetCompare1(TIM9, _pwm); // 修改比较值,修改占空比
   }
\end{lstlisting}

\subsection{风扇自动控制模块}

\begin{lstlisting}[caption={自动调控逻辑}]
 /****************************************
 *         风扇自动控制模块开始          *
 ****************************************/

void auto_control(float temp, float humi) {
  char new_count;

  // 根据温度范围自动调节风扇档位
  if (temp >= TEMP_HIGH) { // 温度 >= 30°C,高速档
    new_count = 3;
    // 只在最高档时响3下蜂鸣器,和手动模式保持一致
    if (buzzer_count < 3) {
      buzzer_tweet();
      buzzer_count++;
    }
    rgb_ctrl(5);                 // 设置RGB指示灯为高温状态
  } else if (temp >= TEMP_MED) { // 温度 >= 25°C,中速档
    new_count = 2;
    buzzer_count = 0;            // 重置蜂鸣器计数
    rgb_ctrl(4);                 // 设置RGB指示灯为中温状态
  } else if (temp >= TEMP_LOW) { // 温度 >= 20°C,低速档
    new_count = 1;
    buzzer_count = 0; // 重置蜂鸣器计数
    rgb_ctrl(3);      // 设置RGB指示灯为低温状态
  } else {            // 温度 < 20°C,关闭风扇
    new_count = 0;
    buzzer_count = 0; // 重置蜂鸣器计数
    rgb_ctrl(2);      // 设置RGB指示灯为正常状态
  }

  // 如果档位发生变化,更新风扇控制和显示
  if (count != new_count) {
    count = new_count;
    update_display(temp, humi);
  }
}
\end{lstlisting}

\begin{lstlisting}[caption={模式匹配逻辑}]
    // 根据当前模式执行风扇控制策略
    if (mode == AUTO_MODE) {
      // 自动模式:根据温度自动调节风扇转速
      auto_control(current_temp, current_humi);
    } else {
      // 手动模式:根据用户设定的档位控制风扇和指示灯
      if (count == 3) { // 最高档位(100%)
        if (buzzer_count < 3) {
          buzzer_tweet(); // 最高档时蜂鸣器响3下提醒
          buzzer_count++;
        }
        rgb_ctrl(5);           // 高速档RGB指示灯
      } else if (count == 2) { // 中档位(66%)
        buzzer_count = 0;      // 重置蜂鸣器计数
        rgb_ctrl(4);           // 中速档RGB指示灯
      } else if (count == 1) { // 低档位(33%)
        buzzer_count = 0;      // 重置蜂鸣器计数
        rgb_ctrl(3);           // 低速档RGB指示灯
      } else if (count == 0) { // 关闭档位(0%)
        buzzer_count = 0;      // 重置蜂鸣器计数
        rgb_ctrl(2);           // 停止档RGB指示灯
      }
    }
\end{lstlisting}

\subsection{风扇手动控制模块}

\begin{lstlisting}
      if (mode == MANUAL_MODE && current_key == K1_PREESED) {
        // K1按键:手动模式下增加风扇档位
        count++;
        buzzer_tweet(); // 按键确认音
        if (count >= ARRAY(fan_pwm))
          count = ARRAY(fan_pwm) - 1; // 限制最大档位
        update_display(current_temp, current_humi);
      } else if (mode == MANUAL_MODE && current_key == K2_PREESED) {
        // K2按键:手动模式下减少风扇档位
        if (count > 0) {
          count--; // 减少档位
        }
        buzzer_tweet(); // 按键确认音
        update_display(current_temp, current_humi);
      }
\end{lstlisting}

\subsection{LCD显示模块}

\begin{lstlisting}[caption={主要信息显示}]
    /****************************************
    *          LCD显示模块开始             *
    ****************************************/

   void update_display(float temp, float humi) {
     char buf[100];

     // 系统关闭状态显示
     if (system_power == 0) {
       sprintf(buf, "---风扇已关闭-----");
       LCDDrawFont16_Next(get_centered_x(buf), 20 + 10 * 7, 4, 320, buf, 0x0000,
                          0xffff);
       LCDDrawFont16_Next(get_centered_x("---按下K4启动---"), 30 + 10 * 8, 4, 320,
                          "---按下K4启动---", 0x0000, 0xffff);
       // 清空其他显示行
       LCDDrawFont16_Next(get_centered_x(""), 30 + 10 * 10, 4, 320, "", 0x0000,
                          0xffff);
       LCDDrawFont16_Next(get_centered_x(""), 30 + 10 * 12, 4, 320, "", 0x0000,
                          0xffff);
       LCDDrawFont16_Next(get_centered_x(""), 30 + 10 * 14, 4, 320, "", 0x0000,
                          0xffff);
       return;
     }

     // 系统运行状态显示
     // 显示风扇转速和档位
     sprintf(buf, "风扇转速: %3d%%(%s)", fan_pwm[count], fan_pwm1[count]);
     LCDDrawFont16_Next(get_centered_x(buf), 20 + 10 * 7, 4, 320, buf, 0x0000,
                        0xffff);

     // 显示当前工作模式
     sprintf(buf, "风扇模式: %s",
             mode == AUTO_MODE ? "-自动-"
                               : (mode == MANUAL_MODE ? "-手动-" : "-自动-"));
     LCDDrawFont16_Next(get_centered_x(buf), 30 + 10 * 8, 4, 320, buf, 0x0000,
                        0xffff);

     // 显示当前温度
     sprintf(buf, "当前温度: %.2f °C", temp);
     LCDDrawFont16_Next(get_centered_x(buf), 30 + 10 * 10, 4, 320, buf, 0x0000,
                        0xffff);

     // 显示当前湿度
     sprintf(buf, "当前湿度: %.1f%%RH", humi);
     LCDDrawFont16_Next(get_centered_x(buf), 30 + 10 * 12, 4, 320, buf, 0x0000,
                        0xffff);
   }
\end{lstlisting}

\begin{lstlisting}[caption={实验标题显示}]
    /*****************************************************
    * 名称:lcd_init()
    * 功能:LCD初始化并打印实验基本信息
    * 参数:name -- 实验名称
    * 返回:无
    ******************************************************/
   void lcd_init(unsigned char name) {
     LCD_DriverInit();
     LCD_Clear(0xFFFF);
     LCD_FillColor(0, 0, 319, 30, 0x4596);
     LCDDrawFnt24(80, 5, "风扇智能调节系统", 0xFFD700, 0x4596);
     LCD_FillColor(0, 240 - 30, 319, 240, 0x4596);
     LCDDrawFnt24(96, 240 - 30 + 5, "嵌入式实践设计", 0xFFD700, 0x4596);
   }
\end{lstlisting}

\subsection{按键控制模块}

\begin{lstlisting}[caption={按键中断初始化}]
    void key_interrupt_init(void) {
        EXTI_InitTypeDef EXTI_InitStructure;
        NVIC_InitTypeDef NVIC_InitStructure;

        RCC_APB2PeriphClockCmd(RCC_APB2Periph_SYSCFG, ENABLE);

        SYSCFG_EXTILineConfig(EXTI_PortSourceGPIOB, EXTI_PinSource12);
        SYSCFG_EXTILineConfig(EXTI_PortSourceGPIOB, EXTI_PinSource13);
        SYSCFG_EXTILineConfig(EXTI_PortSourceGPIOB, EXTI_PinSource14);
        SYSCFG_EXTILineConfig(EXTI_PortSourceGPIOB, EXTI_PinSource15);

        EXTI_InitStructure.EXTI_Line =
            EXTI_Line12 | EXTI_Line13 | EXTI_Line14 | EXTI_Line15;
        EXTI_InitStructure.EXTI_Mode = EXTI_Mode_Interrupt;
        EXTI_InitStructure.EXTI_Trigger = EXTI_Trigger_Falling;
        EXTI_InitStructure.EXTI_LineCmd = ENABLE;
        EXTI_Init(&EXTI_InitStructure);

        NVIC_InitStructure.NVIC_IRQChannel = EXTI15_10_IRQn;
        NVIC_InitStructure.NVIC_IRQChannelPreemptionPriority = 0x02;
        NVIC_InitStructure.NVIC_IRQChannelSubPriority = 0x02;
        NVIC_InitStructure.NVIC_IRQChannelCmd = ENABLE;
        NVIC_Init(&NVIC_InitStructure);
      }

      void EXTI15_10_IRQHandler(void) {
        if (EXTI_GetITStatus(EXTI_Line12) != RESET) {
          EXTI_ClearITPendingBit(EXTI_Line12);
          key_value = K1_PREESED;
          key_interrupt_flag = 1;
        }
        if (EXTI_GetITStatus(EXTI_Line13) != RESET) {
          EXTI_ClearITPendingBit(EXTI_Line13);
          key_value = K2_PREESED;
          key_interrupt_flag = 1;
        }
        if (EXTI_GetITStatus(EXTI_Line14) != RESET) {
          EXTI_ClearITPendingBit(EXTI_Line14);
          key_value = K3_PREESED;
          key_interrupt_flag = 1;
        }
        if (EXTI_GetITStatus(EXTI_Line15) != RESET) {
          EXTI_ClearITPendingBit(EXTI_Line15);
          key_value = K4_PREESED;
          key_interrupt_flag = 1;
        }
      }
\end{lstlisting}

\begin{lstlisting}[caption={部分按键控制模块}]
    /****************************************
     *          按键控制模块开始            *
     ****************************************/

    // K4按键:系统电源开关控制
    if (key_interrupt_flag) {
      delay_count(50); // 按键防抖延时
      char current_key = key_value;
      key_interrupt_flag = 0;

      if (current_key == K4_PREESED) {
        system_power = !system_power; // 切换系统电源状态
        buzzer_tweet();               // 按键确认音
        if (system_power == 0) {
          count = 0;                       // 关闭时重置风扇档位
          fan_pwm_control(fan_pwm[count]); // 停止风扇
          rgb_ctrl(2);                     // 设置RGB为正常状态
        }
        update_display(htu21d_t(), htu21d_h());
        continue;
      }
    }
\end{lstlisting}

\subsection{蜂鸣器模块}

\begin{lstlisting}
    void buzzer_init(void)
    {
      GPIO_InitTypeDef GPIO_InitStructure;

      RCC_AHB1PeriphClockCmd(BUZZER_RCC, ENABLE);

      GPIO_InitStructure.GPIO_Mode = GPIO_Mode_OUT;
      GPIO_InitStructure.GPIO_OType = GPIO_OType_PP;
      GPIO_InitStructure.GPIO_PuPd = GPIO_PuPd_UP;
      GPIO_InitStructure.GPIO_Speed = GPIO_Speed_100MHz;

      GPIO_InitStructure.GPIO_Pin = BUZZER_PIN;
      GPIO_Init(BUZZER_PORT, &GPIO_InitStructure);

      BUZZER_CTRL(OFF);
    }

    void buzzer_tweet(void)
    {
      BUZZER_CTRL(ON);
      delay_ms(5);
      BUZZER_CTRL(OFF);
    }

    void buzzer_stop(void)
    {
      BUZZER_CTRL(OFF);  // 关闭蜂鸣器
    }
\end{lstlisting}

\newpage

\subsection{程序流程图}

\begin{figure}[htbp]
  \centering
  \includegraphics[width=\textwidth]{../figures/program-gra.png}
  \caption{程序流程图}
  \label{fig:1}
\end{figure}