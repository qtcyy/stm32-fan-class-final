\subsection{性能分析}

\qquad 智能风扇控制系统的性能分析从系统瓶颈、资源利用率等多个维度进行评估,为后续优化提供数据支撑。

\subsubsection{性能瓶颈分析}

\textbf{主要性能瓶颈点:}

\textbf{1. 主循环频率过高}
\begin{lstlisting}
while(1) {
    // 主循环执行频率约1000Hz (1ms一次)
    delay_ms(1);
}
\end{lstlisting}

\qquad \textbf{问题分析:}1ms的主循环对MCU资源消耗较大,CPU占用率偏高。\textbf{优化建议:}调整为5-10ms,减少CPU占用率。

\textbf{2. LCD更新频率过高}
\begin{lstlisting}
if (no_motion_timer % 100 == 0) {
    update_display(current_temp, current_humi);  // 每100ms更新一次
}
\end{lstlisting}

\qquad \textbf{问题分析:}LCD更新是耗时操作,每100ms更新过于频繁。\textbf{优化建议:}改为500ms-1s更新一次。

\textbf{3. 温湿度传感器读取频率高}
\begin{lstlisting}
// 系统关闭时仍在读取
send_system_status(htu21d_t(), htu21d_h());  // 每3ms执行一次
\end{lstlisting}

\qquad \textbf{问题分析:}I2C通信耗时,频繁读取影响响应速度。\textbf{优化建议:}缓存数据,降低读取频率。

\textbf{4. 字符串处理开销}
\begin{lstlisting}
int get_centered_x(const char *text) {
    // 每次LCD显示都要计算,包含循环遍历
    for (int i = 0; i < text_len; i++) { ... }
}
\end{lstlisting}

\qquad \textbf{问题分析:}每次显示都重新计算居中位置。\textbf{优化建议:}预计算常用字符串的居中位置。

\subsubsection{性能评估结果}

\begin{table}[H]
  \centering
  \caption{系统性能评估对比表}
  \label{tab:performance_evaluation}
  \begin{tabular}{|c|c|c|c|}
    \hline
    \textbf{性能指标} & \textbf{当前状态} & \textbf{优化目标} & \textbf{改进方案} \\
    \hline
    主循环频率 & 1000Hz & 100Hz & 调整延时为10ms \\
    \hline
    LCD更新频率 & 10Hz & 2Hz & 500ms更新一次 \\
    \hline
    按键响应时间 & 50ms & 10ms & 减少防抖延时 \\
    \hline
    温度读取频率 & 333Hz & 10Hz & 缓存传感器数据 \\
    \hline
    内存使用率 & 中等 & 低 & 优化字符串处理 \\
    \hline
    CPU占用率 & 高 & 中 & 降低循环频率 \\
    \hline
  \end{tabular}
\end{table}

\textbf{优点:}功能完整、逻辑清晰、模块化程度高

\textbf{改进点:}降低循环频率、优化I/O操作频率、减少字符串处理开销

\subsubsection{系统架构性能分析}

\textbf{核心模块组成及性能特征:}

\begin{itemize}
    \vspace{-6pt}
  \item \textbf{系统初始化模块} - 一次性执行,性能影响小
    \vspace{-6pt}
  \item \textbf{按键控制模块} - 中断驱动,响应速度快
    \vspace{-6pt}
  \item \textbf{温湿度监测模块} - I2C通信,存在性能瓶颈
    \vspace{-6pt}
  \item \textbf{风扇控制模块} - PWM硬件输出,性能优秀
    \vspace{-6pt}
  \item \textbf{LCD显示模块} - 刷新频繁,需要优化
    \vspace{-6pt}
  \item \textbf{UART通信模块} - 串口硬件支持,性能良好
\end{itemize}
